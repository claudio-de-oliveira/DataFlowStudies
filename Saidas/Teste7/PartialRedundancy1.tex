\section{Elimina\c{c}\~ao de Redund\^ancias Parciais}


\subsection{Express\~ao Redundante}

Uma express\~ao \'e \emph{redundante} no ponto $p$ se em cada caminho at\'e $p$:
\begin{enumerate}
  \item Ela \'e avaliada antes de alcan\c{c}ar $p$, e
  \item Nenhum de seus operandos constituintes \'e redefinido antes de $p$.
\end{enumerate}

Por exemplo, na Equa\c{c}\~ao~\ref{eq:ExpRed}, as ocorr\^encias de express\~oes em negrito s\~ao redundantes.

\begin{equation}
\centering
  \xymatrix {
    \txt{$a \leftarrow b+c$\\$a \leftarrow \mathbf{b+c}$} \ar[dr] & & \txt{$a \leftarrow b+c$} \ar[dl] \\
                   & \txt{$a \leftarrow \mathbf{b+c}$} \ar[dl] \ar[dr] & \\
    \txt{$a \leftarrow \mathbf{b+c}$} & & \txt{$b \leftarrow 1$\\$a \leftarrow b+c$} 
  }
\label{eq:ExpRed}
\end{equation}

Uma express\~ao \'e \emph{parcialmente redundante} no ponto $p$ se ela \'e redundante ao longo de alguns caminhos, mas n\~ao todos, at\'e $p$.

Por exemplo, na Equa\c{c}\~ao~\ref{eq:ExpParRed}, a express\~ao $b+c$ em negrido no diagrama da esquerda \'e parcialmente redundante.
A inser\c{c}\~ao de uma c\'opia de $b+c$ depois da defini\c{c}\~ao de $b$ pode tornar uma express\c{c}\~ao parcialmente redundante em uma totalmente redundante como mostra o diagrama da direita.

\begin{equation}
\centering
  \xymatrix {
    \txt{$b \leftarrow b+1$} \ar[dr] & & \txt{$a \leftarrow b+c$} \ar[dl] & \txt{$b \leftarrow b+1$\\$a \leftarrow b+c$} \ar[dr] & & \txt{$a \leftarrow b+c$} \ar[dl]\\
     & \txt{$a \leftarrow \mathbf{b+c}$} & & & \txt{$a \leftarrow \mathbf{b+c}$} & \\
  }
\label{eq:ExpParRed}
\end{equation}


\begin{table}[ht]
\centering
\begin{tabular}{l|c|c|c|c|c|c|c|c|c|c|c|c|c|c|c|c|c|c}
	& ENTRY & $B_{1}$ & $B_{2}$ & $B_{3}$ & $B_{4}$ & $B_{5}$ & $B_{6}$ & $B_{7}$ & $B_{8}$ & $B_{9}$ & $B_{10}$ & $B_{11}$ & $B_{12}$ & $B_{13}$ & $B_{14}$ & $B_{15}$ & $B_{16}$ & EXIT \\
\hline
e\_gen & $\{0\}$ & $\{0\}$ & $\{1\}$ & $\{0\}$ & $\{1\}$ & $\{1\}$ & $\{1\}$ & $\{0\}$ & $\{1\}$ & $\{0\}$ & $\{1\}$ & $\{0\}$ & $\{0\}$ & $\{0\}$ & $\{0\}$ & $\{0\}$ & $\{0\}$ & $\{0\}$ \\
e\_kill & $\{0\}$ & $\{1\}$ & $\{0\}$ & $\{1\}$ & $\{0\}$ & $\{0\}$ & $\{0\}$ & $\{0\}$ & $\{0\}$ & $\{0\}$ & $\{0\}$ & $\{0\}$ & $\{0\}$ & $\{0\}$ & $\{0\}$ & $\{0\}$ & $\{0\}$ & $\{0\}$ \\
anticipated\_out & $\{0\}$ & $\{0\}$ & $\{1\}$ & $\{1\}$ & $\{1\}$ & $\{1\}$ & $\{1\}$ & $\{1\}$ & $\{1\}$ & $\{1\}$ & $\{0\}$ & $\{1\}$ & $\{1\}$ & $\{1\}$ & $\{1\}$ & $\{1\}$ & $\{1\}$ & $\{0\}$ \\
anticipated\_in & $\{0\}$ & $\{0\}$ & $\{1\}$ & $\{0\}$ & $\{1\}$ & $\{1\}$ & $\{1\}$ & $\{1\}$ & $\{1\}$ & $\{1\}$ & $\{1\}$ & $\{1\}$ & $\{1\}$ & $\{1\}$ & $\{1\}$ & $\{1\}$ & $\{1\}$ & $\{0\}$ \\
available\_in & $\{0\}$ & $\{0\}$ & $\{0\}$ & $\{0\}$ & $\{0\}$ & $\{1\}$ & $\{1\}$ & $\{1\}$ & $\{1\}$ & $\{1\}$ & $\{1\}$ & $\{1\}$ & $\{1\}$ & $\{1\}$ & $\{1\}$ & $\{1\}$ & $\{1\}$ & $\{1\}$ \\
available\_out & $\{0\}$ & $\{0\}$ & $\{1\}$ & $\{0\}$ & $\{1\}$ & $\{1\}$ & $\{1\}$ & $\{1\}$ & $\{1\}$ & $\{1\}$ & $\{1\}$ & $\{1\}$ & $\{1\}$ & $\{1\}$ & $\{1\}$ & $\{1\}$ & $\{1\}$ & $\{1\}$ \\
earliest & $\{0\}$ & $\{0\}$ & $\{1\}$ & $\{0\}$ & $\{1\}$ & $\{0\}$ & $\{0\}$ & $\{0\}$ & $\{0\}$ & $\{0\}$ & $\{0\}$ & $\{0\}$ & $\{0\}$ & $\{0\}$ & $\{0\}$ & $\{0\}$ & $\{0\}$ & $\{0\}$ \\
 &  &  &  &  &  &  &  &  &  &  &  &  &  &  &  &  &  &  \\
postponable\_in & $\{0\}$ & $\{0\}$ & $\{0\}$ & $\{0\}$ & $\{0\}$ & $\{0\}$ & $\{0\}$ & $\{0\}$ & $\{0\}$ & $\{0\}$ & $\{0\}$ & $\{0\}$ & $\{0\}$ & $\{0\}$ & $\{0\}$ & $\{0\}$ & $\{0\}$ & $\{0\}$ \\
postponable\_out & $\{0\}$ & $\{0\}$ & $\{0\}$ & $\{0\}$ & $\{0\}$ & $\{0\}$ & $\{0\}$ & $\{0\}$ & $\{0\}$ & $\{0\}$ & $\{0\}$ & $\{0\}$ & $\{0\}$ & $\{0\}$ & $\{0\}$ & $\{0\}$ & $\{0\}$ & $\{0\}$ \\
latest & $\{0\}$ & $\{0\}$ & $\{1\}$ & $\{0\}$ & $\{1\}$ & $\{0\}$ & $\{0\}$ & $\{0\}$ & $\{0\}$ & $\{0\}$ & $\{0\}$ & $\{0\}$ & $\{0\}$ & $\{0\}$ & $\{0\}$ & $\{0\}$ & $\{0\}$ & $\{0\}$ \\
used\_out & $\{0\}$ & $\{0\}$ & $\{1\}$ & $\{0\}$ & $\{1\}$ & $\{1\}$ & $\{1\}$ & $\{1\}$ & $\{1\}$ & $\{1\}$ & $\{0\}$ & $\{1\}$ & $\{1\}$ & $\{1\}$ & $\{1\}$ & $\{1\}$ & $\{1\}$ & $\{0\}$ \\
used\_in & $\{0\}$ & $\{0\}$ & $\{0\}$ & $\{0\}$ & $\{0\}$ & $\{1\}$ & $\{1\}$ & $\{1\}$ & $\{1\}$ & $\{1\}$ & $\{1\}$ & $\{1\}$ & $\{1\}$ & $\{1\}$ & $\{1\}$ & $\{1\}$ & $\{1\}$ & $\{0\}$ \\
 &  &  &  &  &  &  &  &  &  &  &  &  &  &  &  &  &  &  \\
$cond\_1$ & $\{0\}$ & $\{0\}$ & $\{1\}$ & $\{0\}$ & $\{1\}$ & $\{0\}$ & $\{0\}$ & $\{0\}$ & $\{0\}$ & $\{0\}$ & $\{0\}$ & $\{0\}$ & $\{0\}$ & $\{0\}$ & $\{0\}$ & $\{0\}$ & $\{0\}$ & $\{0\}$ \\
$cond\_2$ & $\{0\}$ & $\{0\}$ & $\{1\}$ & $\{0\}$ & $\{1\}$ & $\{1\}$ & $\{1\}$ & $\{0\}$ & $\{1\}$ & $\{0\}$ & $\{1\}$ & $\{0\}$ & $\{0\}$ & $\{0\}$ & $\{0\}$ & $\{0\}$ & $\{0\}$ & $\{0\}$ \\
\\
\end{tabular}
\caption{Elimina\c{c}\~ao de Redund\^ancias Parciais --- $((+,\:b,\:c))$}
\end{table}



