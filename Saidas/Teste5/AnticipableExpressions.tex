\section{Disponibilidade de Express\~oes Anticip\'aveis}

A an\'alise \'e para tr\'as (\textsf{backward}) e sua inten\c{c}\~ao \'e determinar em cada ponto do c\'odigo, quais express\~oes podem ser movidas para o in\'icio do bloco (ou para blocos antecedentes).
\begin{itemize}
  \item[$Gen$] Indica quais express\~oes podem ser movidas para o in\'icio do bloco (ou para blocos antecedentes).
  \item[$Kill$] Indica quais express\~oes (considerando o universo inteiro) foram mortas por redefini\c{c}\~oes (anteriores~\footnote{S\'o faz sentido em an\'alises internas ao bloco.}) de seus operandos que ocorrem dentro do bloco.
  \item[$IN$] Indica quais express\~oes podem ser movidas para blocos antecedentes.
  \item[$OUT$] Indica quais express\~oes de blocos subsequ\^entes podem ser movidas para o final do bloco atual -- estas express\~oes podem ou n\~ao serem antecipadas pelo bloco atual.
\end{itemize}

\begin{table}[ht]
\centering
\begin{tabular}{l|l|l|l|l}
	& Gen & Kill & IN & OUT\\
\hline
$B_{1}$ &  $0$ & $1$ & $0$ & $1$\\
$B_{2}$ &  $1$ & $0$ & $1$ & $0$\\
$B_{3}$ &  $0$ & $0$ & $1$ & $1$\\
$B_{4}$ &  $0$ & $0$ & $1$ & $1$\\
\end{tabular}
\caption{Disponibilidade de Express\~oes Anticip\'aveis --- $((+,\:b,\:c))$}
\end{table}

