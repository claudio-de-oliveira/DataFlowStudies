\section{Elimina\c{c}\~ao de Redund\^ancias Parciais}


\subsection{Express\~ao Redundante}

Uma express\~ao \'e \emph{redundante} no ponto $p$ se em cada caminho at\'e $p$:
\begin{enumerate}
  \item Ela \'e avaliada antes de alcan\c{c}ar $p$, e
  \item Nenhum de seus operandos constituintes \'e redefinido antes de $p$.
\end{enumerate}

Por exemplo, na Equa\c{c}\~ao~\ref{eq:ExpRed}, as ocorr\^encias de express\~oes em negrito s\~ao redundantes.

\begin{equation}
\centering
  \xymatrix {
    \txt{$a \leftarrow b+c$\\$a \leftarrow \mathbf{b+c}$} \ar[dr] & & \txt{$a \leftarrow b+c$} \ar[dl] \\
                   & \txt{$a \leftarrow \mathbf{b+c}$} \ar[dl] \ar[dr] & \\
    \txt{$a \leftarrow \mathbf{b+c}$} & & \txt{$b \leftarrow 1$\\$a \leftarrow b+c$} 
  }
\label{eq:ExpRed}
\end{equation}

Uma express\~ao \'e \emph{parcialmente redundante} no ponto $p$ se ela \'e redundante ao longo de alguns caminhos, mas n\~ao todos, at\'e $p$.

Por exemplo, na Equa\c{c}\~ao~\ref{eq:ExpParRed}, a express\~ao $b+c$ em negrido no diagrama da esquerda \'e parcialmente redundante.
A inser\c{c}\~ao de uma c\'opia de $b+c$ depois da defini\c{c}\~ao de $b$ pode tornar uma express\c{c}\~ao parcialmente redundante em uma totalmente redundante como mostra o diagrama da direita.

\begin{equation}
\centering
  \xymatrix {
    \txt{$b \leftarrow b+1$} \ar[dr] & & \txt{$a \leftarrow b+c$} \ar[dl] & \txt{$b \leftarrow b+1$\\$a \leftarrow b+c$} \ar[dr] & & \txt{$a \leftarrow b+c$} \ar[dl]\\
     & \txt{$a \leftarrow \mathbf{b+c}$} & & & \txt{$a \leftarrow \mathbf{b+c}$} & \\
  }
\label{eq:ExpParRed}
\end{equation}


C\'odigo Original:

\begin{table}[ht]
\begin{scriptsize}
\begin{tabular}{l|l|l|l|l|l}
$B_{1}$ & $B_{2}$ & $B_{3}$ & $B_{4}$ & $B_{5}$ & $B_{6}$ \\
\hline
$\{0\}$ & $\{1\}$ & $\{1\}$ & $\{5, 6\}$ & $\{2\}$ & $\{3\}$ \\
$\{3, 2\}$ & $\{5\}$ & $\{6\}$ & $\{7\}$ & $\{4\}$ & $\{4\}$ \\
\hline\
$b:=0$ & $a:=b+c$ & $x:=b+1$ & $a:=b+c$ & $nop$ & $nop$ \\
$if\:0\:goto\:@3$ & $goto\:@5$ &  &  &  &  \\
\end{tabular}
\end{scriptsize}
\end{table}

\begin{scriptsize}
\xy(0, 0)
	*++{\txt{$\mathbf{B_{0}}$\\}}*\frm{-,}="B0";
(0, -15)
	*++{\txt{$\mathbf{B_{1}}$\\$b:=0$\\$if\:0\:goto\:@3$\\}}*\frm{-,}="B1";
(25, -30)
	*++{\txt{$\mathbf{B_{2}}$\\$a:=b+c$\\$goto\:@5$\\}}*\frm{-,}="B2";
(-25, -30)
	*++{\txt{$\mathbf{B_{3}}$\\$x:=b+1$\\}}*\frm{-,}="B3";
(-25, -60)
	*++{\txt{$\mathbf{B_{4}}$\\$a:=b+c$\\}}*\frm{-,}="B4";
(25, -45)
	*++{\txt{$\mathbf{B_{5}}$\\}}*\frm{-,}="B5";
(-25, -45)
	*++{\txt{$\mathbf{B_{6}}$\\}}*\frm{-,}="B6";
(-25, -75)
	*++{\txt{$\mathbf{B_{7}}$\\}}*\frm{-,}="B7";
"B0";"B1" **@{.} ?> *{\dir{>}};
"B1";"B3" **@{.} ?> *{\dir{>}};
"B1";"B2" **@{.} ?> *{\dir{>}};
"B2";"B5" **@{.} ?> *{\dir{>}};
"B3";"B6" **@{.} ?> *{\dir{>}};
"B4";"B7" **@{.} ?> *{\dir{>}};
"B5";"B4" **@{.} ?> *{\dir{>}};
"B6";"B4" **@{.} ?> *{\dir{>}};
\endxy
\end{scriptsize}


C\'odigo Otimizado:

\begin{table}[ht]
\begin{scriptsize}
\begin{tabular}{l|l|l|l|l|l}
$B_{1}$ & $B_{2}$ & $B_{3}$ & $B_{4}$ & $B_{5}$ & $B_{6}$ \\
\hline
$\{0\}$ & $\{1\}$ & $\{1\}$ & $\{5, 6\}$ & $\{2\}$ & $\{3\}$ \\
$\{3, 2\}$ & $\{5\}$ & $\{6\}$ & $\{7\}$ & $\{4\}$ & $\{4\}$ \\
\hline\
$b:=0$ & $t_{5}:=b+c$ & $x:=b+1$ & $a:=t_{5}$ & $nop$ & $t_{5}:=b+c$ \\
$if\:0\:goto\:@3$ & $a:=t_{5}$ &  &  &  & $nop$ \\
 & $goto\:@5$ &  &  &  &  \\
\end{tabular}
\end{scriptsize}
\end{table}

\begin{scriptsize}
\xy(0, 0)
	*++{\txt{$\mathbf{B_{0}}$\\}}*\frm{-,}="B0";
(0, -15)
	*++{\txt{$\mathbf{B_{1}}$\\$b:=0$\\$if\:0\:goto\:@3$\\}}*\frm{-,}="B1";
(25, -30)
	*++{\txt{$\mathbf{B_{2}}$\\$t_{5}:=b+c$\\$a:=t_{5}$\\$goto\:@5$\\}}*\frm{-,}="B2";
(-25, -30)
	*++{\txt{$\mathbf{B_{3}}$\\$x:=b+1$\\}}*\frm{-,}="B3";
(-25, -60)
	*++{\txt{$\mathbf{B_{4}}$\\$a:=t_{5}$\\}}*\frm{-,}="B4";
(25, -45)
	*++{\txt{$\mathbf{B_{5}}$\\}}*\frm{-,}="B5";
(-25, -45)
	*++{\txt{$\mathbf{B_{6}}$\\$t_{5}:=b+c$\\}}*\frm{-,}="B6";
(-25, -75)
	*++{\txt{$\mathbf{B_{7}}$\\}}*\frm{-,}="B7";
"B0";"B1" **@{.} ?> *{\dir{>}};
"B1";"B3" **@{.} ?> *{\dir{>}};
"B1";"B2" **@{.} ?> *{\dir{>}};
"B2";"B5" **@{.} ?> *{\dir{>}};
"B3";"B6" **@{.} ?> *{\dir{>}};
"B4";"B7" **@{.} ?> *{\dir{>}};
"B5";"B4" **@{.} ?> *{\dir{>}};
"B6";"B4" **@{.} ?> *{\dir{>}};
\endxy
\end{scriptsize}


\begin{table}[ht]
\centering
\begin{tabular}{l|c|c|c|c|c|c|c|c}
	& ENTRY & $B_{1}$ & $B_{2}$ & $B_{3}$ & $B_{4}$ & $B_{5}$ & $B_{6}$ & EXIT \\
\hline
e\_gen & $\{00\}$ & $\{00\}$ & $\{10\}$ & $\{01\}$ & $\{10\}$ & $\{00\}$ & $\{00\}$ & $\{00\}$ \\
e\_kill & $\{00\}$ & $\{11\}$ & $\{00\}$ & $\{00\}$ & $\{00\}$ & $\{00\}$ & $\{00\}$ & $\{00\}$ \\
anticipated\_out & $\{00\}$ & $\{10\}$ & $\{10\}$ & $\{10\}$ & $\{00\}$ & $\{10\}$ & $\{10\}$ & $\{00\}$ \\
anticipated\_in & $\{00\}$ & $\{00\}$ & $\{10\}$ & $\{11\}$ & $\{10\}$ & $\{10\}$ & $\{10\}$ & $\{00\}$ \\
available\_in & $\{00\}$ & $\{00\}$ & $\{00\}$ & $\{00\}$ & $\{10\}$ & $\{10\}$ & $\{11\}$ & $\{10\}$ \\
available\_out & $\{00\}$ & $\{00\}$ & $\{10\}$ & $\{11\}$ & $\{10\}$ & $\{10\}$ & $\{11\}$ & $\{10\}$ \\
earliest & $\{00\}$ & $\{00\}$ & $\{10\}$ & $\{11\}$ & $\{00\}$ & $\{00\}$ & $\{00\}$ & $\{00\}$ \\
 &  &  &  &  &  &  &  &  \\
postponable\_in & $\{00\}$ & $\{00\}$ & $\{00\}$ & $\{00\}$ & $\{00\}$ & $\{00\}$ & $\{10\}$ & $\{00\}$ \\
postponable\_out & $\{00\}$ & $\{00\}$ & $\{00\}$ & $\{10\}$ & $\{00\}$ & $\{00\}$ & $\{10\}$ & $\{00\}$ \\
latest & $\{00\}$ & $\{00\}$ & $\{10\}$ & $\{01\}$ & $\{00\}$ & $\{00\}$ & $\{10\}$ & $\{00\}$ \\
used\_out & $\{00\}$ & $\{00\}$ & $\{10\}$ & $\{00\}$ & $\{00\}$ & $\{10\}$ & $\{10\}$ & $\{00\}$ \\
used\_in & $\{00\}$ & $\{00\}$ & $\{00\}$ & $\{00\}$ & $\{10\}$ & $\{10\}$ & $\{00\}$ & $\{00\}$ \\
 &  &  &  &  &  &  &  &  \\
$cond\_1$ & $\{00\}$ & $\{00\}$ & $\{10\}$ & $\{00\}$ & $\{00\}$ & $\{00\}$ & $\{10\}$ & $\{00\}$ \\
$cond\_2$ & $\{00\}$ & $\{00\}$ & $\{10\}$ & $\{00\}$ & $\{10\}$ & $\{00\}$ & $\{00\}$ & $\{00\}$ \\
\\
\end{tabular}
\caption{Elimina\c{c}\~ao de Redund\^ancias Parciais --- $((+,\:b,\:c), (+,\:0,\:1))$}
\end{table}



